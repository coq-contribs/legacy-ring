\documentclass{article}

\usepackage[latin1]{inputenc}
\usepackage[T1]{fontenc}
\usepackage{times}
\usepackage{url}
\usepackage{verbatim}
\usepackage{amsmath}
\usepackage{amssymb}
\usepackage{alltt}
\usepackage{fullpage}
\usepackage[linktocpage,colorlinks]{hyperref}

%%%% Excerpt of the RefMan chapter concerning Legacy Ring %%%%

\input{macros}

\newcommand{\tacindex}[1]{}
\newcommand{\comindex}[1]{}
\newcommand{\errindex}[1]{}
\newcommand{\errindexbis}[2]{}
\newcommand{\ttindex}[1]{}

\title{Legacy implementation of the Ring tactic}
\date{}

\begin{document}

\maketitle

\Warning This tactic is the {\tt ring} tactic of previous versions of
\Coq{} and it should be considered as deprecated. It will probably be
removed in future releases. It has been kept only for compatibility
reasons and in order to help moving existing code to the newer
implementation described above. For more details, please refer to the
Coq Reference Manual, version 8.0.


\subsection{\tt legacy ring \term$_1$ \dots\ \term$_n$
\tacindex{legacy ring}
\comindex{Add Legacy Ring}
\comindex{Add Legacy Semi Ring}}

This tactic, written by Samuel Boutin and Patrick Loiseleur, applies
associative commutative rewriting on every ring.  The tactic must be
loaded by \texttt{Require Import LegacyRing}. The ring must be declared in
the \texttt{Add Ring} command. The ring of booleans (with \texttt{andb}
as multiplication and \texttt{xorb} as addition)
is predefined; if one wants to use the tactic on \texttt{nat} one must
first require the module \texttt{LegacyArithRing}; for \texttt{Z}, do
\texttt{Require Import LegacyZArithRing}; for \texttt{N}, do \texttt{Require
Import LegacyNArithRing}.

The terms \term$_1$, \dots, \term$_n$ must be subterms of the goal
conclusion. The tactic \texttt{ring} normalizes these terms
w.r.t. associativity and commutativity and replace them by their
normal form.

\begin{Variants}
\item \texttt{legacy ring} When the goal is an equality $t_1=t_2$, it
  acts like \texttt{ring\_simplify} $t_1$ $t_2$ and then
  solves the equality by reflexivity.

\item \texttt{ring\_nat} is a tactic macro for \texttt{repeat rewrite
    S\_to\_plus\_one; ring}. The theorem \texttt{S\_to\_plus\_one} is a
  proof that \texttt{forall (n:nat), S n = plus (S O) n}.

\end{Variants}

You can have a look at the files \texttt{LegacyRing.v},
\texttt{ArithRing.v}, \texttt{ZArithRing.v} to see examples of the
\texttt{Add Ring} command.

\subsection{Add a ring structure}

It can be done in the \Coq\ toplevel (No ML file to edit and to link
with \Coq). First, \texttt{ring} can handle two kinds of structure:
rings and semi-rings. Semi-rings are like rings without an opposite to
addition. Their precise specification (in \gallina) can be found in
the file

\begin{quotation}
\begin{verbatim}
LegacyRing_theory.v
\end{verbatim}
\end{quotation}

The typical example of ring is \texttt{Z}, the typical
example of semi-ring is \texttt{nat}.

The specification of a
ring is divided in two parts: first the record of constants
($\oplus$, $\otimes$, 1, 0, $\ominus$) and then the theorems
(associativity, commutativity, etc.).

\begin{small}
\begin{flushleft}
\begin{verbatim}
Section Theory_of_semi_rings.

Variable A : Type.
Variable Aplus : A -> A -> A.
Variable Amult : A -> A -> A.
Variable Aone : A.
Variable Azero : A.
(* There is also a "weakly decidable" equality on A. That means 
  that if (A_eq x y)=true then x=y but x=y can arise when 
  (A_eq x y)=false. On an abstract ring the function [x,y:A]false
  is a good choice. The proof of A_eq_prop is in this case easy. *)
Variable Aeq : A -> A -> bool.

Record Semi_Ring_Theory : Prop :=
{ SR_plus_sym  : (n,m:A)[| n + m == m + n |];
  SR_plus_assoc : (n,m,p:A)[| n + (m + p) == (n + m) + p |];

  SR_mult_sym : (n,m:A)[| n*m == m*n |];
  SR_mult_assoc : (n,m,p:A)[| n*(m*p) == (n*m)*p |];
  SR_plus_zero_left :(n:A)[| 0 + n == n|];
  SR_mult_one_left : (n:A)[| 1*n == n |];
  SR_mult_zero_left : (n:A)[| 0*n == 0 |];
  SR_distr_left   : (n,m,p:A) [| (n + m)*p == n*p + m*p |];
  SR_plus_reg_left : (n,m,p:A)[| n + m == n + p |] -> m==p;
  SR_eq_prop : (x,y:A) (Is_true (Aeq x y)) -> x==y
}.
\end{verbatim}
\end{flushleft}
\end{small}

\begin{small}
\begin{flushleft}
\begin{verbatim}
Section Theory_of_rings.

Variable A : Type.

Variable Aplus : A -> A -> A.
Variable Amult : A -> A -> A.
Variable Aone : A.
Variable Azero : A.
Variable Aopp : A -> A.
Variable Aeq : A -> A -> bool.


Record Ring_Theory : Prop :=
{ Th_plus_sym  : (n,m:A)[| n + m == m + n |];
  Th_plus_assoc : (n,m,p:A)[| n + (m + p) == (n + m) + p |];
  Th_mult_sym : (n,m:A)[| n*m == m*n |];
  Th_mult_assoc : (n,m,p:A)[| n*(m*p) == (n*m)*p |];
  Th_plus_zero_left :(n:A)[| 0 + n == n|];
  Th_mult_one_left : (n:A)[| 1*n == n |];
  Th_opp_def : (n:A) [| n + (-n) == 0 |];
  Th_distr_left   : (n,m,p:A) [| (n + m)*p == n*p + m*p |];
  Th_eq_prop : (x,y:A) (Is_true (Aeq x y)) -> x==y
}.
\end{verbatim}
\end{flushleft}
\end{small}

To define a ring structure on A, you must provide an addition, a
multiplication, an opposite function and two unities 0 and 1.

You must then prove all theorems that make
(A,Aplus,Amult,Aone,Azero,Aeq) 
a ring structure, and pack them with the \verb|Build_Ring_Theory| 
constructor.

Finally to register a ring the syntax is:

\comindex{Add Legacy Ring}
\begin{quotation}
  \texttt{Add Legacy Ring} \textit{A Aplus Amult Aone Azero Ainv Aeq T}
  \texttt{[} \textit{c1 \dots cn} \texttt{].}
\end{quotation}

\noindent where \textit{A} is a term of type \texttt{Set}, 
\textit{Aplus} is a term of type \texttt{A->A->A},
\textit{Amult} is a term of type \texttt{A->A->A},
\textit{Aone} is a term of type \texttt{A},
\textit{Azero} is a term of type \texttt{A},
\textit{Ainv} is a term of type \texttt{A->A},
\textit{Aeq} is a term of type \texttt{A->bool},
\textit{T} is a term of type 
\texttt{(Ring\_Theory }\textit{A Aplus Amult Aone Azero Ainv
  Aeq}\texttt{)}.
The arguments \textit{c1 \dots cn}, 
are the names of constructors which define closed terms: a
subterm will be considered as a constant if it is either one of the
terms \textit{c1 \dots cn} or the application of one of these terms to
closed terms. For \texttt{nat}, the given constructors are \texttt{S}
and \texttt{O}, and the closed terms are \texttt{O}, \texttt{(S O)},
\texttt{(S (S O))}, \ldots

\begin{Variants}
\item \texttt{Add Legacy Semi Ring} \textit{A Aplus Amult Aone Azero Aeq T} 
  \texttt{[} \textit{c1 \dots\ cn} \texttt{].}\comindex{Add Legacy Semi
    Ring}
  
  There are two differences with the \texttt{Add Ring} command: there
  is no inverse function and the term $T$ must be of type
  \texttt{(Semi\_Ring\_Theory }\textit{A Aplus Amult Aone Azero
    Aeq}\texttt{)}.

\item \texttt{Add Legacy Abstract Ring} \textit{A Aplus Amult Aone Azero Ainv 
    Aeq T}\texttt{.}\comindex{Add Legacy Abstract Ring}
  
  This command should be used for when the operations of rings are not
  computable; for example the real numbers of
  \texttt{theories/REALS/}. Here $0+1$ is not beta-reduced to $1$ but
  you still may want to \textit{rewrite} it to $1$ using the ring
  axioms. The argument \texttt{Aeq} is not used; a good choice for
  that function is \verb+[x:A]false+.

\item \texttt{Add Legacy Abstract Semi Ring} \textit{A Aplus Amult Aone Azero
    Aeq T}\texttt{.}\comindex{Add Legacy Abstract Semi Ring}

\end{Variants}

\begin{ErrMsgs}
\item \errindex{Not a valid (semi)ring theory}.
 
  That happens when the typing condition does not hold.
\end{ErrMsgs}

Currently, the hypothesis is made than no more than one ring structure
may be declared for a given type in \texttt{Set} or \texttt{Type}.
This allows automatic detection of the theory used to achieve the
normalization. On popular demand, we can change that and allow several
ring structures on the same set.

The table of ring theories is compatible with the \Coq\ 
sectioning mechanism. If you declare a ring inside a section, the
declaration will be thrown away when closing the section.
And when you load a compiled file, all the \texttt{Add Ring}
commands of this file that are not inside a section will be loaded.

The typical example of ring is \texttt{Z}, and the typical example of
semi-ring is \texttt{nat}. Another ring structure is defined on the
booleans. 

\Warning Only the ring of booleans is loaded by default with the
\texttt{Ring} module. To load the ring structure for \texttt{nat},
load the module \texttt{ArithRing}, and for \texttt{Z},
load the module \texttt{ZArithRing}.

\end{document}
